\section{Conclusion}


Le but de ce projet était donc de permettre la mise à jour simple et rapide d'un Assemblage Catia portant sur une liaison au sol, dimensionnée cinématiquement sur un logiciel annexe nommé Lotus.
En passant par de la programmation VBA, ce résultat est rempli et permet à l'utilisateur une utilisation ergonomique, à l'aide d'un UserForm simple et épurée.
La programmation VBA et les fonctionnalités que nous avons incorporé permettent une automatisation, une répétabilité et une versatilité conséquente.
Vous trouverez ci-dessous les innovations futures de \textit{LASAuto}, que nous n'avons pas pu implémenter :

\paragraph{Développements futurs}
\begin{enumerate}
    \item Faire un parallèle de l'assemblage \texttt{Suspension} avec le module \textsc{Mécamaster}.
    \item Optimiser le rangement des système non symétriques de la voiture (exemple de l'architecture Ackermann pour le système de direction).
    \item Filtrer les points de type \textit{centre de gravité (CG)} dans Lotus.
    \item Enlever les bugs qui apparaissent lorsqu'un Product ne veut pas se mettre à jour.
\end{enumerate}


